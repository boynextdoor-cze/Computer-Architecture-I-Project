%%%%%%%%%%%%%%%%%%%%%%%%%%%%%%%%%%%%%%%%%%%%%%%%%%%%%%%%%%%%%%%%%%%%%
% This is a LaTeX template for students
% working on Homework 6, Computer Architecture I, Spring, 2022.
% You SHALL NOT distribute this template.
%%%%%%%%%%%%%%%%%%%%%%%%%%%%%%%%%%%%%%%%%%%%%%%%%%%%%%%%%%%%%%%%%%%%%
\section{Let's See Some Real World Example}

Each time when you access the course website, your activity will be
recorded into our web server logs! This is the definition of the web
server log for our Computer Architecture course website. Assume our
web server is a 32-bit machine. In this question, we will examine
code optimizations to improve log processing speed. The data
structure for the log is defined below.\\
\textbf{Note: Memory Alignment is considered in Problem 3}
\begin{minted}{c}
    struct log_entry {
        int src_ip;    /* Remote IP address */
        char URL[128]; /* Request URL. You can consider 128 characters are enough. */
        // 4-bytes padding 
        long reference_time; /* The time user referenced to our website. */
        char browser[64]; /* Client browser name */
        int status; /* HTTP response status code. (e.g. 404) */
        // 4-bytes padding 
    } log[NUM_ENTRIES];
\end{minted}
Assume the following processing function for the log. This function
determines the most frequently observed source IPs during the given
hour that succeed to connect our website.

\begin{minted}{c}
    topK_success_sourceIP (int hour);
\end{minted}

\begin{questions}

\question[2] Which field(s) in a log entry will be accessed for the
given log processing function?

{
    \begin{solution}
        Note: There may exist(s) one or more correct choice(s).\\
        % If you find the correct answer, substitute `\choice`
        % by `CorrectChoice`!
        \begin{oneparcheckboxes}
            \CorrectChoice \texttt{src\_ip}
            \choice \texttt{URL}
            \CorrectChoice \texttt{reference\_time}
            \choice \texttt{browser}
            \CorrectChoice \texttt{status}
        \end{oneparcheckboxes}
    \end{solution}
}

\question[1] Assuming 32-byte cache blocks and no prefetching, how
many cache misses per entry does the given function incur on average? \label{q:miss}

{
    \begin{solution}
        % fill your answer in \fillin arguments.
        \fillin[2.25 cache misses][4in]
    \end{solution}
}

\question[3] How can you reorder the data structure to improve
cache utilization and access locality? Justify your modification.

{
    \begin{solution}
        \begin{minted}{c}
struct log_entry {
    int src_ip;    /* Remote IP address */
    int status; /* HTTP response status code. (e.g. 404) */
    long reference_time; /* The time user referenced to our website. */
    char URL[128]; /* Request URL. You can consider 128 characters are enough. */
    char browser[64]; /* Client browser name */
} log[NUM_ENTRIES];
        \end{minted}
        Justification:\\
        Such layout guarantees that there is no padding among any entries and any member variables. For the first time, we access \texttt{src\_ip}, \texttt{status}, \texttt{reference\_time} and the first 16 bytes of \texttt{URL[128]} of the 1st entry. Next, we access the last 16 bytes of \texttt{browser[64]} of the 1st entry, \texttt{src\_ip}, \texttt{status} and \texttt{reference\_time} of the 2nd entry, with \texttt{URL[128]} and \texttt{browser[64]} of the 2nd entry being ignored. Next time the access starts at the beginning of the 3rd entry. After that, the procedure described above will be executed over and over again, with 2 entries being accessed within 2 misses. Therefore, \textbf{the average misses per entry is 1.}
    \end{solution}
}


\question[6] To mitigate the miss in the question \ref{q:miss},
design a different data structure. How would you rewrite the
program to improve the overall performance?

Your answer shall include:

\begin{itemize}
    \item A new layout of data structure of our server logs.
    \item A description of how your function would improve the
    overall performance.
    \item How many cache misses per entry does your improved
    design incur on average?
\end{itemize}

{
    \begin{solution}
        %%%%%%%%%%%%%% YOUR ANSWER HERE %%%%%%%%%%%%%%%%%%%%%%%%%
        \\
        \\
        Layout:
        \begin{minted}{c}
struct log_entry_useful {
    int src_ip;    /* Remote IP address */
    int status; /* HTTP response status code. (e.g. 404) */
    long reference_time; /* The time user referenced to our website. */
} log[NUM_ENTRIES];
        \end{minted}
        \begin{minted}{c}
struct log_entry_useless {
    char URL[128]; /* Request URL. You can consider 128 characters are enough. */
    char browser[64]; /* Client browser name */
} log[NUM_ENTRIES];
        \end{minted}

        Description:\\
        If we separate the accessed fields and unaccessed fields into two parts, the overall performance can be highly improved. Since the total size of \texttt{log\_entry\_useful} is only 16 bytes and the block size is 32 bytes, two entries can be loaded once there is a cache miss. Therefore, only consider the accessed fields, after the first cache miss 6 fields in total can be loaded to the cache, so all the following 5 references (the remaining accesses of two adjacent entries) will incur cache hits, which performs much better than the original layout or the reordered layout, whose data structure contains useless but large fields to prevent successive cache hits.\\
        \\
        Calculation:\\
        There is only 1 cache miss out of 6 consecutive cache references (in terms of field), namely two consecutive entry accesses. So \textbf{the average cache misses per entry is 0.5}.

        %%%%%%%%%%%%%%%%%%%%%%%%%%%%%%%%%%%%%%%%%%%%%%%%%%%%%%%%%
    \end{solution}
}

\end{questions}